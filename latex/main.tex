%%%% Opcje
%% wmii --- Wydział Matematyki i informatyki
%%
%% Kierunek:
%%      inf --- Informatyka
%%
%% Poziom studiów (praca):
%%      mgr --- magisterska




\documentclass[wmii, inf, mgr]{uwmthesis}

\usepackage[utf8]{inputenc}
\usepackage[MeX]{polski}
\usepackage{graphicx}
\usepackage{url}
\usepackage{float}

\usepackage{listings}
\lstset{language=csh,
  showspaces=false,
  showtabs=false,
  breaklines=true,
  showstringspaces=false,
  breakatwhitespace=true,
  escapeinside={(*@}{@*)},
  basicstyle=\ttfamily,
  columns=fullflexible
}%% dla listingów kodu



\date{2022}
\title{Uczenie zespołowe, zastosowanie i porównania efektywności wybranych algorytmów\\ w dostępnych zbiorach danych.}
\author{Andrzej Strzeszewski}
\etitle{Ensemble learning,\\ application and performance \\comparisons of selected algorithms\\ on available datasets}
\wykonanaw{Katedra Metod Matematycznych Informatyki}
\ewykonanaw{the Chair of Mathematical Methods of Computer Science}

\podkierunkiem{dr. Andrzeja Jankowskiego}
\epodkierunkiem{Andrzej Jankowski, PhD}

\begin{document}

\maketitle





\begin{streszczenie}
Praca zawiera opisy algorytmów uczenia maszynowego. Celem pracy jest połączenie tych algorytmów  metodami uczenia zespołowego oraz sprawdzenie czy dane rozwiązanie jest dokładniejsze. Dla rozwiązań liczone będą błędy na podstawie różnych mierników. Takich jak: Mean Absolute Error, Mean Absolute Percentage Error. Rozważania zakończyłem AB-testami, w których z odpowiednim poziomem ufności stwierdziłem czy uczenie zespołowe daje większą efektywność niż pojedyncze algorytmy.  

\end{streszczenie}

\begin{abstract}
dfsf 

\end{abstract}


\tableofcontents

\chapter*{Wstęp}

sdfsdfs

\chapter{Wymagania aplikacji}

sgsgfsdg

\chapter{Użyte technologie}

fgxf

\chapter{Źródła danych }
dasdsa

\chapter{Wizualizacja danych }
dasdsa

\chapter{Trenowanie modeli }
dasdsa


\chapter{Opis wykorzystanych algorytmów }
(tutaj jeszcze nie wiem jakich użyję więc wrzucam jak najwięcej, żeby mieć potem w czym wybierać)
\section{Mierzenie dokładności algorytmów decyzyjnych}

tutaj coś o błędach

\section{Regresja liniowa}

dfghdfgh

\section{knn}

dfghdfgh


\section{Sieć neuronowa}

dfghdfgh


\section{Drzewo decyzyjne – regresyjne}

dfghdfgh

\section{Regresyjny las losowy (ang. Regression forest)}

fghfgh

\section{Drzewo decyzyjne wzmocnione (ang. Boosted decision tree)}

ghjghj


hjkgk




\chapter{Implementacja i omówienie kodu}


\section{Algorytmy}
fudtuy


\section{Łączenie algorytmów}
fudtuy
\subsection{Bagging}
dsa

\subsection{Boosting}


\chapter{Sprawdzanie efektywności uczenie zaespołowego- AB testy}

cgjh

\chapter{Podsumowanie}

gh


\listoffigures
dsa

\listoftables
sdadsa

\chapter{Spis algorytmów}
sdadsa

\chapter{Indeks stosowanych oznaczeń}
sdadsa

\chapter{Używane źródła danych i dokumentów z Internetu}
\textit{Geographic coordinate system}
\url{https://en.wikipedia.org/wiki/Geographic_coordinate_system}, dostęp online: 20.01.2022.


\thebibliography{1}
\bibitem{bib1}
\textit{Geographic coordinate system}
 \url{https://en.wikipedia.org/wiki/Geographic_coordinate_system}, dostęp online: 20.01.2022.


\end{document}
